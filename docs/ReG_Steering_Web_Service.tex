\documentclass[a4paper]{article}
\usepackage{graphicx}
\usepackage{url}
\usepackage[small, bf]{caption}

% A4 is 211mm wide x 297
%\setlength {\topmargin}{-20mm}
%\setlength {\headsep}{0mm}
%\setlength {\headheight}{0mm}
%\setlength {\textheight}{255mm} %29.5cm is a4
%\setlength {\textwidth}{169mm}
\setlength {\captionmargin}{20pt}
%\oddsidemargin -7mm
%\evensidemargin 0mm

\begin{document}

\title{The RealityGrid Steering Web Service.}

\author{A.~R.~Porter \\
Manchester Computing,\\University of Manchester,\\Oxford Road,\\
Manchester,\\M13 9PL.}

%\date{July, 2005}

\maketitle

%--------------------------------------------------------------
% Change log
\begin{table}
\begin{center}
\begin{tabular}{l|l|p{5cm}|c}
\hline\hline
Version & Author & Notes & Date \\
\hline
0.1 & A.~R.~Porter & Initial draft --- work in progress alongside implementation & 06/07/2005\\
\hline\hline
\end{tabular}
\end{center}
\end{table}

\pagebreak

\tableofcontents

\pagebreak
%--------------------------------------------------------------

\begin{section}{Introduction}
The RealityGrid Steering Web Service is a replacement for the Open
Grid Services Infrastructure (OGSI)-based RealityGrid Steering Grid
Service (SGS).  As of January 2004, the OGSI has been superceded by
the Web Services Resource Framework (WSRF).  In response to this, we
have taken the lessons learned from implementing and using the SGS and
created a new, WSRF-based version which we term a Steering Web Service
(SWS).  This document describes the interface presented by this
service.
\end{section}

%------------------------------------------------------------------

\begin{section}{Pre-requisites}
The SWS is implemented in Perl and uses the WSRF::Lite package by Mark
Mc Keown which may be obtained from
\url{http://www.sve.man.ac.uk/Research/AtoZ/ILCT}.  In order to run
the WSRF::Lite container you will need Perl (either version 5.6 or
one more recent than version 5.8.0).  You will also need SOAP::Lite version
0.65\_5 or higher plus associated pre-requisites.  Note that this
version of SOAP::Lite does not yet seem to be available from CPAN and
must therefore be installed manually.

The application to be steered must be instrumented with version 2.0a or
greater of the RealityGrid steering library.
\end{section}

%------------------------------------------------------------------

\begin{section}{The ResourceProperties Document.}
\label{sec:RPDoc}

At the heart of a WS-Resource is the ResourceProperties document which
provides a standard way of storing state within the service.  The
ResourceProperties document of the SWS contains the elements listed in
table~\ref{tab:resourceProps}.  WSRF provides standard methods for
interacting with the ResourceProperties document which we describe
below with examples from Perl using the WSRF::Lite and SOAP::Lite
packages.  The RealityGrid steering library version 2.0 and greater
also includes gSoap bindings for these methods (with the exception
of GetResourceProperty which has so far proved problematic).

\begin{table}
\begin{center}
\begin{tabular}{l|p{6cm}|l|c}
\hline\hline
ResourceProperty   & Description & Type & Writable\\
\hline
applicationName    & The name of the application being steered & string & Y\\
applicationStatus  & Status of the application being steered & string & Y\\
checkpointEPR      & The EPR of the checkpoint from which job is currently running & string & N\\
chkTypeDefinitions & The ChkTypes registered by the app & XML & Y\\
clientCount        & How many steering clients are attached & integer & N\\
controlMsg         & Control messages from the steering client & Array, XML & Y\\
creationTime       & Date and time that the service was created & string & N\\
dataSource         & Document giving the EPRs and IOType labels providing data to the app. associated with this service & XML & Y\\ 
inputFileContent   & Contents of the job's main input file (if any) & string & N\\

ioTypeDefinitions  & The IOTypes registered by the app & XML & Y \\
lastModifiedTime   & Time (since epoch) that ResourceProperty doc was last modified & integer & N\\
latestStatusMsg    & The most recently received status message from the app & XML & N\\
machineAddress     & IP address of the machine on which application is running & string & Y\\
maxNumControlMsg   & Maximum number of control messages buffered by SWS & integer & N\\
maxNumStatusMsg    & Maximum number of status messages buffered by SWS & integer & N\\
maxRunTime         & Maximum wall-clock time for job will run (minutes) & integer & Y\\
paramDefinitions   & The parameters registered by the app & XML & Y\\
registryEPR        & EPR of the registry where this service is registered & string & N\\
ServiceGroupEntry  & EPR of the service modelling this service's entry in the registry & string & N\\
statusMsg          & Status messages from the app & Array, XML & Y\\
steererStatus      & Status of the steering client (if any) & string & Y\\
supportedCommands  & The steering commands supported by the app & XML & Y\\
workingDirectory   & The directory in which the app is running & string & Y\\
\hline\hline
\end{tabular}
\end{center}
\caption{The ResourceProperties of the SWS.  EPR stands for End-Point
Reference and refers to the address at which a web service may be
contacted.}
\label{tab:resourceProps}
\end{table}

\begin{subsection}{GetResourceProperty}
Use this to retrieve a single ResourceProperty (RP) from the service.  Using
Perl and WSRF::Lite, this looks like:
\begin{verbatim}
$ans = WSRF::Lite
       -> uri(`http://www.sve.man.ac.uk/SWS')
       -> wsaddress(WSRF::WS_Address->new()->Address($target))
       -> GetResourceProperty( 
             SOAP::Data->value(`applicationStatus')->type(`xml') );
\end{verbatim}
where \texttt{\$target} is the location (also known as the End-Point
Reference or EPR) of the service being queried and
\texttt{`applicationStatus'} is the RP being retrieved.
\end{subsection}

\begin{subsection}{GetMultipleResourceProperties}
As for GetResourceProperty but can retrieve more than one property in
a single call.  The method takes a small XML document describing which
ResourceProperties to get.  To retrieve two ResourceProperties using
Perl and WSRF::Lite:
\begin{verbatim}
$searchTerm = "<wsrp:ResourceProperty xmlns:wsrp=\"$WSRF::Constants::WSRP\">".
              "applicationStatus</wsrp:ResourceProperty>";
$searchTerm .= "<wsrp:ResourceProperty xmlns:wsrp=\"$WSRF::Constants::WSRP\">".
               "steererStatus</wsrp:ResourceProperty>";

$ans= WSRF::Lite
       -> uri($WSRF::Constants::WSRP)
       -> wsaddress(WSRF::WS_Address->new()->Address($target))
       -> GetMultipleResourceProperties( $header, 
                              SOAP::Data->value($searchTerm)->type('xml') );
\end{verbatim}

\end{subsection}

\begin{subsection}{GetResourcePropertyDocument}
This is a very simple method --- it takes no arguments and returns the
complete ResourcePropertyDocument of the service.  Note that the
ResourcePropertyDocument may also be obtained by simply executing an
http `get' operation on the endpoint of the service (\textit{e.g.}\ by
pasting the EPR of the SWS into a web-browser and hitting \texttt{Return}).
\end{subsection}

\begin{subsection}{SetResourceProperties}
This method can be used to Insert, Update or Delete a
ResourceProperty.  Insert is used to add to array-based
ResourceProperties while Update and Delete are self-explanatory.
ResourceProperties can be configured (within the service code) so that
they cannot be deleted or updated.  Using Perl and WSRF::Lite, a call
to this method looks something like:
\begin{verbatim}
my $insertTerm =`<wsrp:Insert><applicationStatus>'.
                `NOT_STARTED</applicationStatus></wsrp:Insert>';

$ans =  WSRF::Lite
       -> uri($uri)
       -> wsaddress(WSRF::WS_Address->new()->Address($target))  
       -> SetResourceProperties( 
               SOAP::Data->value( $insertTerm )->type( `xml' ) );
\end{verbatim}
\end{subsection}

\end{section}

%---------------------------------------------------------------------

\begin{section}{Using the SWS for steering}
\label{sec:sws_steering}

In this section we will walk through the actions that must be taken in
order to steer an application via the SWS.

\begin{subsection}{Discovering the application}
To be implemented but ServiceGroups in WSRF are very similar to those
in OGSI.
\end{subsection}

\begin{subsection}{Attaching to the application}
Client calls the \texttt{Attach} method (no arguments) on the SWS.
Returns an XML document describing the commands that the application
supports.  This document conforms to the RealityGrid steering schema.
\end{subsection}

\begin{subsection}{Getting the parameter definitions}
Once successfully attached to the SWS, the next stage is to get the
information published by the application.  The most important aspect
of this is the various monitored and steerable parameters supported by
the application.  This information is held on the SWS in the {\bf
paramDefinitions} ResourceProperty and therefore can be accessed using
some of the methods described in section~\ref{sec:RPDoc}.  The
information itself is an XML document conforming to the RealityGrid
steering schema.
\end{subsection}

\begin{subsection}{Getting the IOType and ChkType definitions}
As with the parameter definitions, the IOType and ChkType definitions
are also held as ResourceProperties ({\bf ioTypeDefinitions} and {\bf
chkTypeDefinitions}, respectively) on the SWS and therefore can be
accessed using the standard methods of section~\ref{sec:RPDoc}.
Again, the definitions are in the form of XML documents conforming to
the RealityGrid steering schema.  Note that all of these definitions
may be obtained in one go by getting the complete ResourceProperties
document of the service.  However, be aware that there may be a delay
in an application actually setting some of these ResourceProperties
and that the Steering API permits a program to change these
definitions during execution.  It is therefore essential to check for
updates --- see section~\ref{sec:notification}.
\end{subsection}

\begin{subsection}{Getting status messages from the application}
Once a client has successfully obtained and parsed the parameter
definitions published by the application, it is ready to check for
status messages.  There are two ways of doing this.  The simplest is
to get the latest status message received by the SWS from the
application.  This is held in the {\bf latestStatusMsg} ResourceProperty and
is best obtained (especially on lightweight clients) by calling the
\texttt{GetResourceProperty} method.  The second, and more robust way of
obtaining status messages is to access the {\bf statusMsg} ResourceProperty.
This uses an array to buffer up to {\bf maxNumStatusMsg} (another
ResourceProperty) status messages.  Once the array is full, the
arrival of a new status message causes the oldest message held in the
buffer to be replaced.  Calling \texttt{GetResourceProperty} for this
ResourceProperty returns an XML document containing an array of status
messages in the form:
\begin{verbatim}
          <ResourceProperty>
            <statusMsg>
              <ReG_steer_message UID=`1'>
                <App_status>
                  ...
                </App_status>
              </ReG_steer_message>
              <ReG_steer_message UID=`2'>
                <App_status>
                  ...
                </App_status>
              </ReG_steer_message>
            </statusMsg>
          </ResourceProperty>
\end{verbatim}

As of version $2.0$ of the steering library, all steering messages are
given a unique identifer specified as an atrribute with name `Msg\_UID' on
the ReG\_steer\_message element.  This enables a client to check to see
whether it has seen the message before which is particularly useful
when getting the  {\bf latestStatusMsg} ResourceProperty.  This is because
this property is {\em only} changed when the application sends another
status message to the SWS --- it is unaffected by a client reading its
value.
\end{subsection}

\begin{subsection}{Sending control messages to the application}
In order to send a control message to the application a client must
construct a message conforming to the RealityGrid steering schema and
send it to the SWS by using the \texttt{SetResourceProperty} method to
set the {\bf controlMsg} ResourceProperty. {\it e.g.} To send a Stop
command the document that the client must supply as the argument to
the SetResourceProperty call would be:
\begin{verbatim}
          <wsrp:Insert>
            <controlMsg>
              <ReG_steer_message>
                <Steer_control>
                  <Command>
                    <Cmd_name>STOP</Cmd_name>
                  </Command>
                </Steer_control>
              </ReG_steer_message>
            </controlMsg>
          </wsrp:Insert>
\end{verbatim}
\end{subsection}

\begin{subsection}{Checking for `notifications'}
\label{sec:notification}
It is important that a client check for notifications from the steered
application in order that it can take action when, for example, the
set of parameters registered by the application changes.  Support for
this feature is included in the SWS through the {\bf lastModifiedTime}
ResourceProperty which records the last time (in the form of a single
integer holding the number of seconds since the UNIX epoch) that the
ResourceProperty Document was modified.  It is important to note that
this parameter is {\em not} changed by the setting of either the {\bf
statusMsg} or {\bf controlMsg} ResourceProperties since the frequency
with which those two properties are likely to be set would limit its
usefulness.

In contrast to the SGS, the SWS does not explicitly notify any
attached steering clients that the application has finished running.
Instead, once the application has finished ({\it i.e.} has called
Steering\_finalize or caught a signal) the SWS unregisters itself from
the registry and then expires.  Thus, any attached steering client is
effectively notified of the application's completion by the
disappearance of the SWS.
\end{subsection}

\begin{subsection}{Detaching from the application}
The steering client notifies the SWS that it has finished steering the
application by calling the \texttt{Detach} method.  Unlike the SGS, the
SWS does not explicitly acknowledge this call.  (However, the value of
the {\bf clientCount} ResourceProperty will be decremented by one upon
a successful detach.)  If there are no steering clients attached to
the SWS then the application ceases to emit status messages to it.
\end{subsection}

\end{section}

%---------------------------------------------------------------------

\begin{section}{The Methods of the SWS}

Table~\ref{table:SWSmethods} lists all of the methods supported by the
SWS.

\begin{table}
\begin{center}
\begin{tabular}{l|p{4cm}|p{4cm}}
\hline\hline
Method & Description & Returns\\
\hline
Attach & Attach to the application & XML doc containing supported commands\\
Detach & Detach from the application & --\\
\hline
Destroy & Destroy the SWS and remove any associated registry entry & --\\
GetResourceProperty & Get a single RP & Value of the RP \\
GetResourcePropertyDocument & Get the whole ResourceProperty doc & The RP doc\\
GetMultipleResourceProperties & Get one or more RPs & Value(s) of the RP(s) \\
SetResourceProperties & Set the value of one or more RPs & -- \\
\hline\hline
\end{tabular}
\end{center}
\caption{The methods of the SWS.  Those in the lower section of the table 
are standard WS-RF methods (only the most useful of these are listed).}
\label{table:SWSmethods}
\end{table}

\end{section}

\end{document}

